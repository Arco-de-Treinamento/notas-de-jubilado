\section{Introdução}

Nesse capítulo abordaremos os conjuntos numéricos e suas operações básicas. Com frequência, os conjuntos numéricos são utilizados para representar coleções de elementos que possuem alguma ordem, seja ela quantidade de elementos ou outros valores. Exemplos de conjuntos numéricos incluem o conjunto dos números naturais $\naturais$, o conjunto dos inteiros $\inteiros$, o conjunto dos números reais $\reais$, entre outros.

Essas coleções de elementos são a base da matemática moderna e suas operações são amplamente utilizadas em diversos campos da matemática, como a teoria da computação, a teoria da probabilidade e a estatística. Ao logo desse capítulo estudaremos esses tópicos em detalhes e como são utilizados no desenvolvimento de aplicações.