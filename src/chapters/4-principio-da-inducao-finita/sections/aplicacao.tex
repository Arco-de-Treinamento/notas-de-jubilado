\section{O Princípio da Indução Finita na computação}

O \emph{Princípio da Indução Finita} pode ser visto com frequência em provas de correção de algoritmos e estruturas de dados recursivas, como o algoritmo de Fibonacci ou o algoritmo de busca binária, que podem ser verificados por indução finita para garantir que eles funcionam corretamente para todos os tamanhos de entrada.

Tomando o algoritmo de ordenação \textbf{mergesort} como exemplo, iremos provar que o algoritmo ordena corretamente um array de tamanho $n$.

\begin{itemize}
    \item \emph{Caso base} [$n = 1$]: Isso é trivial, pois um array com apenas um elemento por definição já está ordenado;
    
    \item \emph{Passo indutivo}: Assumimos que a propriedade é verdadeira para um array de tamanho \textbf{n = k}, e então provaremos que é verdadeira para \textbf{n = k + 1}.
\end{itemize}

Considere um array \textbf{arr} de tamanho \textbf{k + 1} e que será utilizado para provar que o array resultante \textbf{sortedArr} também é ordenado em ordem crescente. Para isso, mostraremos que para qualquer par de índices $i$ e $j$ onde $0 <= i < j < k+1$, temos:

\begin{align*}
    sortedArr[i] <= sortedArr[j]
\end{align*}

\begin{codesnip}{Exemplo em JavaScript - Mergesort}{JavaScript}
const arr = [47, 71, 51, -45, 3, 1, -93, -12];

function mergeSort(arr) {
    // Verifica se o array possui meio
    if (arr.length < 2) return arr;
    // Econtra o meio e cria dois novos arrays partindo do original
    const middle = Math.floor(arr.length / 2);
    const arrLeft = arr.slice(0, middle);
    const arrRight = arr.slice(middle, arr.length);
    // Retorna o array ordenado
    return merge(mergeSort(arrLeft), mergeSort(arrRight));
}

function merge(arrLeft, arrRight) {
    let leftIndex = 0;
    let rightIndex = 0;
    const result = [];
    // Percorre os dois arrays e adiciona o menor valor a result
    while (leftIndex < arrLeft.length && 
            rightIndex < arrRight.length) {
    if (arrLeft[leftIndex] < arrRight[rightIndex]) {
        result.push(arrLeft[leftIndex]);
        leftIndex++;
    } else {
        result.push(arrRight[rightIndex]);
        rightIndex++;
        }
    }
    // Retorna o array result concatenado com o arrLeft e arrRight
    return result
    .concat(arrLeft.slice(leftIndex))
    .concat(arrRight.slice(rightIndex));
}

const sortedArr = mergeSort(arr);
console.log(sortedArr);    
// Resultado: 93, -45, -12, 1, 3, 47, 51, 71
\end{codesnip}

\begin{itemize}
    \item \emph{Se $i$ e $j$ estiverem ambos em \textbf{arrLeft} ou em \textbf{arrRight}}: Então a propriedade de ordenação segue da hipótese de indução;
    
    \item \emph{Se $i$ e $j$ estiverem um em \textbf{arrLeft} e outro em \textbf{arrRight}}: Então \textbf{sortedArr[i] e sortedArr[j]} são, respectivamente, o menor elemento não processado em arrLeft e arrRight. Portanto, \textbf{sortedArr[i] <= arrLeft[x] e sortedArr[j] <= arrRight[y]} para algum x e y. Como arrLeft e arrRight são ordenados, temos \textbf{arrLeft[x] <= arrRight[y]}, e portanto, \textbf{sortedArr[i] <= arrLeft[x] <= arrRight[y] <= sortedArr[j]}.
\end{itemize}

Dessa forma, provamos por indução finita que o algoritmo mergeSort sempre retorna um array ordenado em ordem crescente.

Este exemplo foi criado em JavaScript. Para mais informações, acesse: \link{https://developer.mozilla.org/pt-BR/docs/Web/JavaScript}{JavaScript}. \\




