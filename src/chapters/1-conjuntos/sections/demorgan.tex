\subsection{Leis de De Morgan}

As \emph{Leis de De Morgan} são duas propriedades fundamentais que relacionam conjuntos e seus complementares. Inicialmente, temos que, dado os conjuntos $A$ e $B$, o complementar de $A \uniao B$ é igual a interseção do complementar de ambos os conjuntos:

\[
    (A \uniao B)\complementar = A\complementar \inter B\complementar.
\]

Já no segundo caso temos que o complementar $A \inter B$ é o mesmo que a união de ambos os conjuntos:

\[
    (A \inter B)\complementar = A\complementar \uniao B\complementar.
\]


Alguns autores ainda trazem as \emph{Leis de De Morgan} como a transformação de conjunções (operador lógico "E") e disjunções (operador lógico "OU") em conjunções e disjunções de conjuntos complementares.



