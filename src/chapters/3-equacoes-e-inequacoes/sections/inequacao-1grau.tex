\section{Inequação do 1° grau}

\emph{A inequação do 1° grau} é uma desigualdade algébrica que, assim como a equação do 1° grau não possui nenhuma potência em sua incógnita. Ao determinar sua incógnita podemos encontrar uma faixa de valores que satisfazem a desigualdade.

Uma desigualdade de grau 1 é representada por uma expressão matemática que contém sinais de desigualdade, tais como $<$ (menor que), $>$ (maior que), $\leq$  (menor ou igual a) ou $geq$ (maior ou igual a), podendo possui as seguintes formas:

\begin{align*}
    & ax+b <0;\\
    & ax+b >0;\\
    & ax+b \le 0;\\
    & ax+b \ge 0;
\end{align*}


Sejam $a, b, c, d \in \R$; $n \in \N^*$, as \emph{inequações do 1° grau} podem possuir as seguintes propriedades:

\begin{enumerate}[i.]
    \item Invariância por adição de números reais: $a < b \implies a+c < b+c$;
    \item Invariância por multiplicação de números reais positivos:  $a < b ; c>0 \implies a \cdot c < b \cdot c$;
    \item Mudança por multiplicação de números reais negativos: $a < b ; c<0 \implies a \cdot c > b \cdot c$;
    \item Se $a < b$, então $\frac 1 a > \frac 1 b$, para $a \e b$ ambos positivos, ou ambos negativos;
    \item Se $a,b \geq 0$ e $c>0$, segue que $a < b \implies a^c < b^c$;
    \item Se $a,b < 0$ e $n$ par, segue que $a < b \implies a^n > b^n$;
    \item Se $a,b < 0$ e $n$ ímpar, segue que $a < b \implies a^n < b^n$;
    \item Se $a< b$ e $c< d$, então $a+c < b+d$;
    \item Para $a, b, c, d \in \R_+$. Se $a< b$ e $c< d$, então $ac < bd$.
\end{enumerate}

Os resultados são análogos para os tipos $>$, $\leq$ e $\geq$.
