\subsection{Soma e produto}

O método da \emph{soma e produto} permite econtrar as raízes de uma equação do 2° grau de modo prático apenas se, e somente se,

\begin{align*}
    x_1 + x_2 = \dfrac {-b} a \; \e \; x_1 * x_2 = \dfrac c a.
\end{align*}

Assim sendo:

\begin{align*}
    x_1 + x_2 &= \frac{ -b + \sqrt{\Delta}}{2a} + \frac{ -b - \sqrt{\Delta}}{2a} \implies\\
    x_1 + x_2 &= \frac{ -b + \sqrt{\Delta} -b - \sqrt{\Delta}}{2a} \implies\\
    x_1 + x_2 &= \frac{-2b}{2a} \implies\\
    x_1 + x_2 &= \frac{-b}{a}
\end{align*}

Além disso, temos que:
\begin{align*}
    x_1 * x_2 &= \frac{ -b + \sqrt{\Delta}}{2a} * \frac{ -b - \sqrt{\Delta}}{2a} \implies\\
    x_1 * x_2 &= \frac{ (-b)^2 -b\sqrt{\Delta} + b\sqrt{\Delta} -(\sqrt{\Delta})^2}{2a} \implies\\
    x_1 * x_2 &= \frac{b^2 - \Delta}{4a^2} \implies\\
    x_1 * x_2 &= \frac{b^2 - \prn{b^2 - 4ac}}{4a^2} \implies\\
    x_1 * x_2 &= \frac{b^2 - b^2 + 4ac}{4a^2} \implies \\
    x_1 * x_2 &= \frac{4ac}{4a^2} \implies \\
    x_1 * x_2 &= \frac{c}{a}
\end{align*}

